\documentclass{beamer}

\title{Introduction to GDB}
\subtitle{(or how to debug like a pro)}
\author{Mistral Contrastin}

\beamertemplatenavigationsymbolsempty

\setbeamerfont{page number in head/foot}{size=\small}
\setbeamertemplate{footline}[frame number]
\begin{document}
  \frame{\maketitle}

  \begin{frame}
    \begin{center}
      {\LARGE ``What is wrong with printf?'' -- \textsl{Radu}} \\
    \end{center}
  \end{frame}

  \begin{frame}{GDB as a Swiss Army Knife}
    \begin{itemize} 
      \item Debugger (duh)
      \item Memory inspector
      \item Disassembler
      \item Runtime
      \item Patcher
      \item Hex editor
      \item \ldots
    \end{itemize}
  \end{frame}

  \begin{frame}{Start with the basics}
    \begin{description}[leftmargin=2em,style=nextline]
      \item[Compilation] \texttt{\$ gcc -g <file>}
      \item[Start] \texttt{\$ gdb <binary>}
      \item[Get to \texttt{main}] \texttt{> break main}
      \item[Inspect code] \texttt{> list}
      \item[Execute] \texttt{> run}
      \item[Run line by line] \texttt{> next} / \texttt{> step}
      \item[Inspect variable] \texttt{> print var}
    \end{description}
  \end{frame}

  \begin{frame}{Unleash the power of \textbf{breakpoints} and \textbf{watchpoints}}
    
    Lets you stop a program at arbitrary lines or at even deeper granularity.

    \begin{description}
      \item[Set] \texttt{> break [filename:]<line number> [if expr]} \\
                 \texttt{> watch [filename:]<line number> [if expr]}
      \item[Conditional] \texttt{> cond <breakpoint number> expr}
      \item[List] \texttt{> info breakpoints}
      \item[Delete] \texttt{> delete <breakpoint number>}
      \item[Temporary] \texttt{> tbreak <line number>}
      \item[Function] \texttt{> break <function name>}
      \item[Instruction] \texttt{> break *<memory address>}
    \end{description}
  \end{frame}

  \begin{frame}{Inspect memory}
    \begin{description}
      \item[Inspect locals] \texttt{> info local}
      \item[Inspect arguments] \texttt{> info args}
      \item[Stack trace] \texttt{> where / backtrace / info stack}
      \item[Stack trace limit] \texttt{> set backtrace limit <n>}
    \end{description}
  \end{frame}

  \begin{frame}{Inspect memory, cont'd}
    General memory inspection format is \texttt{p x/nfu}, where n is
    repeat count, f is display format and u is the unit size.
    \vfill

    \begin{columns}
      \column{0.5\textwidth}
      Options for display format
      \begin{description}
        \item[bytes] b
        \item[halfwords] h
        \item[words] w
        \item[giant words] g
      \end{description}

      \column{0.5\textwidth}
      Common display formats
      \begin{description}
        \item[decimal] d
        \item[hexadecimal] x
        \item[octal] o
        \item[string] s
        \item[instruction] i
      \end{description}
    \end{columns}

  \end{frame}

  \begin{frame}{Text User Interface}
    \begin{columns}
      \column{0.6\textwidth}
        \begin{description} 
          \item[Enter] \texttt{C-x a} within GDB or \\ \texttt{\$ gdb -tui}
          \item[Exit] \texttt{C-x a} 
          \item[Single key] \texttt{C-x s}
        \end{description}
    
      \column{0.4\textwidth}
        Single key mode bindings:
        \begin{description}
          \item[run] r
          \item[continue] c
          \item[next] n
          \item[step] s
          \item[where] w
          \item[info locals] v
          \item[exit] q
        \end{description}
    \end{columns}
  \end{frame}

  \begin{frame}{Use \textbf{core files} to save time}
    \begin{itemize}
      \item \texttt{\$ ulimit -c <number of bytes>} 
      \item Linux dumps at the current working directory with name \texttt{core}
      \item For better core file name on Linux (\texttt{<filename>\_<pid>.core})

        \texttt{\$ sudo su}

        \texttt{\$ echo \%e\_\%p.core > /proc/sys/kernel/core\_pattern}
      \item Mac dumps at \texttt{/cores} with name \texttt{core.<PID>}
      \item Task manager \texttt{>} Right click \texttt{>} Create Dump File \\
        {\scriptsize (please use Cygwin instead)}
      \item \texttt{gdb <file> <core>}
    \end{itemize}
  \end{frame}

  \begin{frame}{C debugging tips and tricks}
    \begin{itemize}
      \item Principle of Confirmation
      \item ``Premature optimization is the root of all evil.'' \\
        \hfill -- \textsl{Donald Knuth}
      \item $\uparrow$ Except when you use it as a lynt
      \item Always compile with -Wall -Wextra -pedantic
    \end{itemize}
  \end{frame}

  \begin{frame}{Go down the rabbit hole}
    \begin{itemize}
      \item Attaching processes
      \item Remote debugging 
      \item Kernel debugging
      \item Hardware and memory breakpoints
      \item Multiple TUI windows
      \item LLDB
    \end{itemize}
  \end{frame}

  \begin{frame}
    \begin{center}
      {\LARGE Thank you!}
    \end{center}
  \end{frame}
\end{document}
